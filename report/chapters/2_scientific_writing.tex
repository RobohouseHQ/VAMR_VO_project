\chapter{Scientific Writing}\label{chap:scientific_writing}

This chapter gives you some tips on how to write scientifically. It should prevent you from making the most common mistakes people do and help you with creating a well written report.

\section{General Style}

\begin{itemize}
	\item A report/paper is not a short-story. There is no build-up to a climax. The climax should be in the abstract. Even better, in the title.
	\item Hierarchical exposition, not linear: this goes in hand with the previous point.
	A hierarchical exposition means that you start with the core of your work (The main thing your project was about) and then go into details in following sections.
	Do not build up to the core of your work with too much background/preliminaries as it would be the case in a linear exposition.
	\item At the beginning of every chapter/major section, you should summarize what the content of the section will be.
	A person should get a good sense of the report by reading the first paragraph of each section.
	\item Express your thoughts succinctly.
	Avoid unnecessary words or phrases and be precise and specific.
	\item Definitions are useful if they are used often.
	Do not define something if it is only used once.
	\item Be generous with your references.
	Do not compare your results with others by pointing the deficiencies of their work; rather, state how your results are adding to the body of knowledge others have created.
	\item Notation is extremely important.
	Good notation facilitates understanding. You do not want the reader to mentally perform translations every time they see a symbol.
\end{itemize}

\section{Important Stuff}

\begin{itemize}
	\item Use active verb tense whenever possible: instead of \emph{An analysis of the signal noise is performed using a discrete Fourier transform.} write \textbf{We perform an analysis of the signal noise using a discrete Fourier transform.}
	\item Make short sentences with one statement. 
	Long sentences with multiple statements are complicated and hard to understand. 
	Write to be understood, not to impress!
	\item Be concrete/specific: instead of \emph{We use a model to predict the state} write \textbf{We use a linear model of the attitude dynamics to predict the quadrotor's state at time $t + \Delta t$}.
	\item Be precise: instead of \emph{We assume the model to be linear}, say \textbf{We design a linear model of the system dynamics}. (You assume the \emph{system dynamics} to be linear and hence you create a linear model.)
	\item Be consistent: this basically applies to every level. Denote the same thing always with the same word, create figures with a similar style, etc.
	\item Do not make unsubstantiated statements.
	Do not use \emph{It is common knowledge} or \emph{Several researchers have shown}.
	Instead use constructs like \textbf{Recently, several researchers~\cite{KleinMurray2007,Strasdat2010WhyFilter} have shown}.
\end{itemize}

\section{Small Things}

\begin{itemize}
	\item Do not use \emph{don't}, \emph{aren't}, etc., use \textbf{do not} and \textbf{are not}.
	\item Do not use words like \emph{simply}, \emph{highly}, \emph{just}, \emph{very}, \emph{a lot}, etc.
	\item Use \textbf{because} instead of \emph{due to the fact that}, \textbf{to} instead of \emph{in order to}, etc.
	\item When referencing to figures, sections, etc., use capital letters: see Figure~\ref{img:notation}, see Section~\ref{chap:scientific_writing}.
	\item Every figure, table, and algorithm must be referenced in the text.
	\item Put punctuation marks after each formula as if they were text. 
	Separate multiple consecutive formulas by commas and put a period if you start a new sentence after the formula.
	For more details, see Section~\ref{sec:math}.
	\item Avoid brackets. If something is important enough to be mentioned it does not need brackets; if not, it does not need to be mentioned at all.
	\item In English, after a colon (:) you continue with small letters.
	\item Use \emph{we} to refer to yourself: \textbf{We} developed an algorithm to ...
	\item Do not use \emph{ours}.
	\item Use the ``Oxford comma'' when you list three or more items, e.g., we used red, green, and blue balls.
	\item Put a hyphen for multi-word adjectives, e.g., high-speed robotics, state-of-the-art research.
	\item Put details in an appendix.
	\item Avoid single-sentence paragraphs.
	\item Do \textbf{not} start sentences with a citation (\emph{\cite{KleinMurray2007} proposed a similar approach.}) or a variable (\emph{$f$ is the focal length.}).
	\item Number all equations.
	\item Do not use the word ``equation'' before a reference to an equation, unless it is at the beginning of a sentence.
	Example: Equation (12) is a simplification of (4).
\end{itemize}